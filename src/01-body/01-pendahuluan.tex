\chapter{\babSatu}

\section{Latar Belakang}

Setiap tahun Forum Manajemen Indonesia (FMI) mengadakan seminar terbuka yang
diikuti oleh ribuan peserta dari berbagai daerah di Indonesia. FMI mengelola
acara tersebut menggunakan beberapa sistem yang tidak terintegrasi. Beberapa
diantaranya adalah pendaftaran peserta yang menggunakan formulir online lalu
melakukan konfirmasi pembayaran dengan mengirim email atau menghubungi contact
person. Setelah pembayaran divalidasi, panitia akan mengirimkan email secara
manual kepada pendaftar mengenai informasi pembayaran dan acara seminar.
Peserta dapat meminta informasi lain mengenai acara melalui email atau kontak
person. Sertifikat peserta yang dibagikan setelah acara berlangsung dibuat
manual dan dicetak dalam bentuk fisik.

Pada tahun 2020 terjadi pandemi yang membuat FMI tidak dapat mengadakan seminar
sehingga FMI memilih webinar sebagai alternatif dari seminar. Dengan perubahan
tersebut FMI perlu melakukan beberapa keputusan mengenai pelaksanaan kegiatan
tersebut. Salah satunya adalah sistem yang dapat mengelola webinar. Dibutuhkan
sistem terintegrasi yang mampu menangani pengelolaan dan beberapa fitur
otomatis yang dapat mempermudah dan mengurangi resiko pelaksanaan acara.

Aplikasi manajemen yang diperlukan adalah aplikasi yang dapat menangani
penyediaan informasi dan pendaftaran webinar, menyediakan buku tamu ketika
acara berlangsung, dan distribusi fasilitas webinar misalkan sertifikat
peserta. Sistem diharapkan dapat digunakan dalam jangka waktu lama untuk
kegiatan selanjutnya dan modular sehingga fitur baru dapat ditambahkan dengan
mudah. Selain itu, diperlukan laporan dan rangkuman tentang data yang terkumpul
seperti demografi peserta, minat peserta, dan data pembayaran pada webinar
sehingga dapat menjadi salah satu dasar pengambilan keputusan. Laporan yang
dibangun diharapkan dapat memberikan hasil berupa gambaran detail tentang
pelaksanaan webinar.

\section{Identifikasi Masalah}

Berdasarkan latar belakang masalah di atas, maka dapat diidentifikasi masalah
sebagai berikut:

\begin{ol}
  \item Perlu sistem terintegrasi untuk untuk mengelola pelaksanaan webinar.
  \item Beberapa kegiatan yang berulang pada pelaksanaan webinar perlu diotomatisasi.
  \item Dibutuhkan solusi yang lebih mudah untuk peserta mengakses informasi yang dibutuhkan.
  \item Sistem yang modular sehingga dapat menyesuaikan kebutuhan dalam jangka panjang.
\end{ol}


\section{Rumusan Masalah}

Dari latar belakang masalah dan identifikasi masalah tersebut dapat dirumuskan
masalah yakni “Bagaimana membangun aplikasi yang modular dan terintegrasi
sehingga dapat digunakan untuk mengelola webinar dalam jangka panjang”.
Khususnya pada organisasi Forum Manajemen Indonesia (FMI).

\begin{ol}
\item Bagaimana membuat sistem yang terintegrasi untuk mengelola pelaksanaan webinar?
\item bagaimana cara agar kegiatan yang berulang dapat diotomatisasi?
\item Bagaimana cara agar peserta dengan mudah mengakses fasilitas dan informasi webinar?
\item Bagaimana cara membuat sistem yang modular dan dapat menyediakan kebutuhan webinar dalam jangka panjang?
\end{ol}

\section{Batasan Masalah}

Batasan masalah dalam penelitian ini adalah :

\begin{ol}
  \item Sistem dapat mengelola webinar mulai dari informasi tentang webinar, pendaftaran peserta, validasi pembayaran, memberikan notifikasi mengenai pelaksanaan webinar, dan distribusi fasilitas webinar.
  \item Data bersumber dari peserta berupa data pendaftaran dan pembayaran serta data dari penyelenggara acara seperti informasi webinar, daftar member, dan sertifikat peserta.
  \item Aplikasi dibangun dan diimplementasikan untuk mengelola acara webinar FMI.
  \item Aplikasi berjalan pada web dan disimulasikan menggunakan Chrome desktop (versi 86.0) pada aplikasi operasi Windows 10 Pro dan Ubuntu 20.04.
  \item Dibangun menggunakan bahasa pemrograman Python dan database PostgreSQL.
  \item Aplikasi dipasang pada virtual server dengan sistem operasi Linux.
  \item Menggunakan Pycharm sebagai Integrated Development System (IDE) dalam proses pembangunan.
  \item Aplikasi dipisahkan menjadi backend dan frontend. Backend berupa Single Page Application (SPA) dan frontend dibangun menggunakan template engine bernama QWEB.
  \item Laporan disajikan dalam bentuk PDF, pivot, dan grafik.
  \item Kartu peserta dan sertifikat ditampilkan dalam bentuk SVG dan di render pada canvas untuk proses pengunduhan.
\end{ol}

\section{Tujuan Penelitian}

Tujuan yang ingin dicapai pada penelitian ini adalah:

\begin{ol}
  \item Menyediakan aplikasi manajemen webinar yang mengintegrasikan komponen pelaksanaan webinar sehingga data mudah diakses.
  \item Mengurangi kesalahan data yang dikelola dengan membuat sistem validasi.
  \item Membangun fitur pencarian yang lengkap dan mudah sehingga dapat aplikasi dapat menampilkan data sesuai yang dibutuhkan.
  \item Menyediakan laporan tentang data yang terkumpul dari webinar sebagai sumber informasi tentang kondisi webinar yang sedang berlangsung.
\end{ol}

\section{Manfaat dan Kegunaan Penelitian}

Dengan adanya penelitian ini, diharapkan aplikasi dapat mengelola
penyelenggaraan webinar terutama pada Forum Manajemen Indonesia (FMI) dengan
efektif dan efisien. Webinar banyak dipilih sebagai alternatif seminar untuk
menaati protokol kesehatan pada masa pandemi Covid-19. Selain itu aplikasi yang
dibangun memungkinkan untuk digunakan pada tahun-tahun berikutnya untuk
menyelesaikan permasalahan terkait manajemen seminar.

\begin{ol}
  \item Manfaat bagi Forum Manajemen Indonesia:
    \begin{ol}
      \item Mempermudah pengelolaan webinar.
      \item Menghindari resiko kesalahan data.
      \item Mempercepat proses pelaksanaan webinar.
      \item Mendapatkan laporan tentang pelaksanaan webinar.
    \end{ol}
  \item Manfaat bagi peserta webinar:
    \begin{ol}
      \item Dapat mendaftar webinar dengan lebih mudah.
      \item Fasilitas webinar dapat diakses dengan mudah dan darimana saja.
      \item Sertifikat partisipasi peserta dapat diunduh dan divalidasi langsung.
      \item Informasi webinar dan partisipan dapat diakses dengan mudah.
    \end{ol}
  \item Manfaat bagi peneliti:
    \begin{ol}
      \item Memahami proses pembangunan perangkat lunak.
      \item Memberikan pengalaman dalam penerapan pada dunia nyata.
      \item Memenuhi salah satu syarat kelulusan program sarjana (S1).
    \end{ol}
  \item Manfaat bagi Universitas:
    \begin{ol}
      \item Penelitian dapat dijadikan referensi pembangunan aplikasi manajemen webinar.
      \item Dapat dijadikan sebagai referensi penerapan aplikasi manjemen yang terintegrasi.
      \item Menambah arsip universitas tentang perancangan aplikasi monitoring server.
    \end{ol}
\end{ol}

\section{Metode Penelitian}

Teknik penelitian yang dipakai adalah deskriptif kuantitatif dengan fokus pada
penerapan rekayasa perangkat lunak dengan mempertimbangkan respon pengguna
terhadap aplikasi. Penerapan dilakukan pada kasus nyata dengan data riil.
Penelitian menggunakan prosedur waterfall sesuai alur implementasi yang telah
disusun. Data responden dikumpulkan pada akhir penggunaan aplikasi melalui
formulir survei untuk meningkatkan pengalaman pengguna kedepannya.

\section{Jadwal Penelitian}

Dibawah ini tabel jadwal penelitian :

\begin{table}
  \centering
  \caption{Jadwal Penelitian}
  \label{tab:tab1}
  \begin{tabular}{| c | c | c | c | c | c | c | c |}
    \hline
    \multirow{2}{*}{\bo{Nomor}} & \multirow{2}{*}{\bo{Kegiatan}} & \multicolumn{6}{c|}{\bo{Bulan}} \\
    \cline{3-8}
    & & \bo{Ke-1} & \bo{Ke-2} & \bo{Ke-3} & \bo{Ke-4} & \bo{Ke-5} & \bo{Ke-6} \\
    \hline
    1 & Pengajuan Proposal & X & - & - & - & - & - \\
    2 & Pengumpulan Data & - & X & X & - & - & - \\
    3 & Analisis dan perancangan & - & - & X & - & - & - \\
    4 & Implementasi Rancangan & - & - & - & X & - & - \\
    5 & Pengujian Aplikasi & - & - & - & - & X & - \\
    6 & Pengumpulan Data & - & - & - & X & X & X \\
    \hline
  \end{tabular}
\end{table}


\section{Sistematika Penulisan Laporan}
Skripsi ini terdiri 5 bab dengan pokok bahasan tiap bab sebagai berikut :

\bo{Bab I Pendahuluan}

\setlength{\leftskip}{4em} \indent Dalam bab I akan dibahas mengenai Latar
Belakang Masalah, Identifikasi Masalah, Pembatasan Masalah, Rumusan Masalah,
Tujuan Penelitian, Kegunaan Penelitian, Metode Penelitian dan Sistematika
Penelitian.

\setlength{\leftskip}{2em}
\bo{Bab II Tinjauan Pustaka}

A. Teori Pendukung.

\setlength{\leftskip}{4em} \indent Berisi teori-teori yang mendukung penerapan
aplikasi manajemen webinar antara lain: masalah tentang manajemen webinar,
teori tentang pembangunan aplikasi berbasis website, dan pemrograman Python.

\setlength{\leftskip}{2em}
\bo{Bab III Analisis dan Desain Sistem}

A. Analisa Sistem

\setlength{\leftskip}{4em}
1. Flowchart Sistem

2. Perancangan Alur Aplikasi

3. Pembentukan Aturan atau Rule

\setlength{\leftskip}{2em}
B.  Perancangan Diagram Pengguna

C.  Perancangan Desain Tampilan

\setlength{\leftskip}{2em}
\bo{Bab IV Implementasi Dan Hasil}

A. Tampilan Antarmuka

B.	Pengujian Aplikasi

C.	Analisa Hasil Implementasi Aplikasi

\setlength{\leftskip}{2em}
\bo{Bab V Penutup}

A. Kesimpulan

\setlength{\leftskip}{4em}
Dikemukakan pokok-pokok penelitian sesuai rumusan masalah dan tujuan penelitian.


