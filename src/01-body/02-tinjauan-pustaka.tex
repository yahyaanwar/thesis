\chapter{\babDua}

\section{Landasan Teori}


\subsection{Seminar dan Webinar}
\subsubsection{Seminar}
\paragraph{Definisi Seminar}
Seminar pada umumnya merupakan sebuah bentuk pengajaran akademis, baik di sebuah universitas maupun diberikan oleh suatu organisasi komersial atau profesional. Kata seminar berasal dari kata Latin seminarum, yang berarti "tanah tempat menanam benih" (Rinayanthi, I. Nengah Laba \& Ni Made, 2018).

Sebuah seminar biasanya memiliki fokus pada suatu topik yang khusus, di mana mereka yang hadir dapat berpartisipasi secara aktif. Seminar sering kali dilaksanakan melalui sebuah dialog dengan seorang moderator seminar, atau melalui sebuah presentasi hasil penelitian dalam bentuk yang lebih formal. Biasanya, para peserta bukanlah seorang pemula dalam topik yang didiskusikan (di universitas, kelas-kelas seminar biasanya disediakan untuk mahasiswa yang telah mencapai tingkatan atas). Sistem seminar memiliki gagasan untuk lebih mendekatkan mahasiswa kepada topik yang dibicarakan. Di beberapa seminar dilakukan juga pertanyaan dan debat. Seminar memiliki sifat lebih informal dibandingkan sistem kuliah di kelas dalam sebuah pengajaran akademis (Ilham Prastya, 2021).

\paragraph{Keunggulan Seminar}
\subsubsection{Webinar}
\paragraph{Definisi Webinar}
Webinar ini merupakan gabungan dua kata, yaitu web dan seminar. Dengan demikian, pengertian webinar adalah suatu kegiatan pertemuan (seminar) yang dilakukan secara online (virtual) dengan memanfaatkan jaringan internet dan dapat diikuti oleh beberapa orang dari berbagai lokasi yang berbeda (Rury Yuliatri, 2020).

Solusi yang dapat dilakukan sebagai alternatif seminar dalam kondisi ini adalah webinar. Webinar merupakan singkatan dari web seminar yaitu pertemuan yang dilakukan secara daring dengan memanfaatkan jaringan internet. Kegiatan webinar ini dilakukan secara daring dengan menggunakan berbagai aplikasi yang dapat menghubungkan partisipan dilokasi yang berbeda-beda. Alternatif ini dapat diterapkan pada berbagai bidang mulai dari kegiatan bisnis, pembelajaran, hingga diskusi \citep{fachrunnisa}.

\paragraph{Keunggulan Webinar}
Beberapa kelebihan dari webinar ini adalah praktis bagi partisipan dalam mengikutinya. Partisipan webinar tidak perlu keluar rumah untuk menghadiri webinar (Imelda Rahma, 2020). Peserta juga tidak perlu berkumpul dengan banyak orang. Secara otomatis hal ini akan mengurangi dampak partisipan terkena Covid-19.

\subsection{Pandemi Covid-19}
\subsubsection{Definisi Pandemi}
\subsubsection{Definisi Covid-19}
Coronavirus adalah kumpulan virus yang bisa menginfeksi sistem pernapasan. Pada banyak kasus, virus ini hanya menyebabkan infeksi pernapasan ringan, seperti flu. Namun, virus ini juga bisa menyebabkan infeksi pernapasan berat, seperti infeksi paru-paru (pneumonia) (Merry Dame Cristy Pane, 2021).

Virus ini menular melalui percikan dahak (droplet) dari saluran pernapasan, misalnya ketika berada di ruang tertutup yang ramai dengan sirkulasi udara yang kurang baik atau kontak langsung dengan droplet (Merry Dame Cristy Pane, 2021).

Selain virus SARS-CoV-2 atau virus Corona, virus yang juga termasuk dalam kelompok ini adalah virus penyebab Severe Acute Respiratory Syndrome (SARS) dan virus penyebab Middle-East Respiratory Syndrome (MERS). Meski disebabkan oleh virus dari kelompok yang sama, yaitu coronavirus, COVID-19 memiliki beberapa perbedaan dengan SARS dan MERS, antara lain dalam hal kecepatan penyebaran dan keparahan gejala (Merry Dame Cristy Pane, 2021).

\subsubsection{Sejarah Pandemi Covid-19}
Infeksi virus Corona disebut COVID-19 (Corona Virus Disease 2019) dan pertama kali ditemukan di kota Wuhan, China pada akhir Desember 2019. Virus ini menular dengan sangat cepat dan telah menyebar ke hampir semua negara, termasuk Indonesia, hanya dalam waktu beberapa bulan (Merry Dame Cristy Pane, 2021).


\subsection{Metode Survey ......../metode development}
\subsubsection{Definisi Metode}
% \paragraph{Definisi Seminar}
% \paragraph{Keunggulan Seminar}
% \subsubsection{Webinar}
% \paragraph{Definisi Webinar}
% \paragraph{Keunggulan Webinar}


\subsection{Perangkat Lunak yang Digunakan}
\subsubsection{OpenERP}
\paragraph{Definisi OpenERP}
\paragraph{Sejarah OpenERP}
\subsubsection{Python}
\paragraph{Definisi Python}
\paragraph{Sejarah Python}
\subsubsection{Postgresql}
\paragraph{Definisi Postgresql}
\paragraph{Keunggulan Postgresql}
\subsubsection{Chrome Web Browser}
\paragraph{Definisi Chrome Web Browser}
\paragraph{Sejarah Chrome Web Browser}
\subsubsection{Pycharm}
\paragraph{Definisi Pycharm}
\paragraph{Keunggulan Pycharm}


\section{Kajian Pustaka}
Penelitian terkait sistem manajemen webinar menjadi topik bahasan yang menarik dikarenakan webinar adalah alternatif seminar yang banyak diminati terutama pada masa pandemi seperti yang sedang berlangsung. Masih jarang ditemukan aplikasi yang khusus untuk mengelola penyelenggaraan webinar.

Penelitian sebelumnya mengenai pembangunan aplikasi webinar dilakukan oleh (Hafid Yoza Putra, dkk. 2020) dengan judul “Pembangunan Aplikasi Web dan Mobile Sistem Informasi Webinar di Era New Normal”. Penelitian ini membahas tentang pembangunan sistem informasi webinar pada masa pandemi menggunakan metode waterfall dengan objek penelitian berupa kegiatan sebelumnya yang berupa seminar menggunakan framework Laravel.

Penelitian yang hampir sama juga dilakukan oleh (Arman Suryadi Karim, dkk. 2019) dengan judul “Pembangunan Sistem Informasi Manajemen Seminar (Nasional dan Internasional) pada IBI Darmajaya”. Penelitian ini berupa rancangan sistem serta metode pembangunan aplikasi manajemen seminar menggunakan metode Prototyping-Based Methodology. Prototyping-Based Methodology adalah salah satu metodologi yang biasa digunakan dalam Object-Oriented Systems Analysis and Design (OOSAD). Komponen sistem dalam penelitian ini dimodelkan menggunakan Unified Modelling Language (UML). Tools yang biasa digunakan antara lain Use case diagram, Activity diagram, Sequence diagram dan Class diagram.

Lebih lama lagi penelitian oleh (Andi Nugroho, 2006) dengan judul “Aplikasi Web Informasi dan Registrasi Peserta Seminar, Workshop, Talkshow pada Acara Seminar Nasional Pengamplikasian Telematika (SINAPTIKA)” yang mempunyai topik yang lebih general dengan menampilkan informasi tentang kegiatan dan memanfaatkan Google Form sebagai formulir pendaftaran.


\section{Desain Sistem (Perancangan)}

\subsection{Kebutuhan Data}
\subsubsection{Data Input}
\subsubsection{Gambaran Proses}
\subsubsection{Data Output}

\subsection{Desain Sistem (Arsitektur)}
\subsubsection{Kebutuhan Sistem}
\subsubsection{Struktur Sistem}
\paragraph{View}
\paragraph{Model}
\paragraph{Controller}

\subsection{Desain Database}
\subsubsection{Data Dalam Sistem}
\subsubsection{Diagram Relasi Entiti}

\subsection{Desain Menu Aplikasi}
\subsubsection{Pendaftaran}
\subsubsection{Pembayaran}
\subsubsection{Upload Dokumen}
\subsubsection{Halaman Administrasi}


% \subsection{Desain Input}

% \subsubsection{Formulir Registrasi Individu}
% \subsubsection{Formulir Registrasi Kolektif}

% \subsection{Desain Input}

% \subsubsection{Alur Proses}
% \subsubsection{Desain Output}
